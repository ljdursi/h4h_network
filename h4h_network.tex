\documentclass[twoside,sfsidenotes,notoc]{tufte-book}

\hypersetup{colorlinks}% uncomment this line if you prefer colored hyperlinks (e.g., for onscreen viewing)

%%
% Book metadata
\title{The HPC4Health Network}
\subtitle{Building an Ontario-wide platform for human health data}
\author{HPC4Health: High-Performance Data and Computing for Health Care}
\publisher{HPC4Health}

%%
% If they're installed, use Bergamo and Chantilly from www.fontsite.com.
% They're clones of Bembo and Gill Sans, respectively.
%\IfFileExists{bergamo.sty}{\usepackage[osf]{bergamo}}{}% Bembo
%\IfFileExists{chantill.sty}{\usepackage{chantill}}{}% Gill Sans

%\usepackage{microtype}

%%
% Just some sample text
\usepackage{lipsum}

%%
% For nicely typeset tabular material
\usepackage{booktabs}

%%
% For graphics / images
\usepackage{graphicx}
\setkeys{Gin}{width=\linewidth,totalheight=\textheight,keepaspectratio}
\graphicspath{{graphics/}}

% The fancyvrb package lets us customize the formatting of verbatim
% environments.  We use a slightly smaller font.
\usepackage{fancyvrb}
\fvset{fontsize=\normalsize}

%%
% Prints argument within hanging parentheses (i.e., parentheses that take
% up no horizontal space).  Useful in tabular environments.
\newcommand{\hangp}[1]{\makebox[0pt][r]{(}#1\makebox[0pt][l]{)}}

%%
% Prints an asterisk that takes up no horizontal space.
% Useful in tabular environments.
\newcommand{\hangstar}{\makebox[0pt][l]{*}}

%%
% Prints a trailing space in a smart way.
\usepackage{xspace}

%%
% Some shortcuts for Tufte's book titles.  The lowercase commands will
% produce the initials of the book title in italics.  The all-caps commands
% will print out the full title of the book in italics.
\newcommand{\vdqi}{\textit{VDQI}\xspace}
\newcommand{\ei}{\textit{EI}\xspace}
\newcommand{\ve}{\textit{VE}\xspace}
\newcommand{\be}{\textit{BE}\xspace}
\newcommand{\VDQI}{\textit{The Visual Display of Quantitative Information}\xspace}
\newcommand{\EI}{\textit{Envisioning Information}\xspace}
\newcommand{\VE}{\textit{Visual Explanations}\xspace}
\newcommand{\BE}{\textit{Beautiful Evidence}\xspace}

\newcommand{\TL}{Tufte-\LaTeX\xspace}

% Prints the month name (e.g., January) and the year (e.g., 2008)
\newcommand{\monthyear}{%
  \ifcase\month\or January\or February\or March\or April\or May\or June\or
  July\or August\or September\or October\or November\or
  December\fi\space\number\year
}


% Prints an epigraph and speaker in sans serif, all-caps type.
\newcommand{\openepigraph}[2]{%
  %\sffamily\fontsize{14}{16}\selectfont
  \begin{fullwidth}
  \sffamily\large
  \begin{doublespace}
  \noindent\allcaps{#1}\\% epigraph
  \noindent\allcaps{#2}% author
  \end{doublespace}
  \end{fullwidth}
}

% Inserts a blank page
\newcommand{\blankpage}{\newpage\hbox{}\thispagestyle{empty}\newpage}

\usepackage{units}

% Typesets the font size, leading, and measure in the form of 10/12x26 pc.
\newcommand{\measure}[3]{#1/#2$\times$\unit[#3]{pc}}

% Macros for typesetting the documentation
\newcommand{\hlred}[1]{\textcolor{Maroon}{#1}}% prints in red
\newcommand{\hangleft}[1]{\makebox[0pt][r]{#1}}
\newcommand{\hairsp}{\hspace{1pt}}% hair space
\newcommand{\hquad}{\hskip0.5em\relax}% half quad space
\newcommand{\TODO}{\textcolor{red}{\bf TODO!}\xspace}
\newcommand{\ie}{\textit{i.\hairsp{}e.}\xspace}
\newcommand{\eg}{\textit{e.\hairsp{}g.}\xspace}
\newcommand{\na}{\quad--}% used in tables for N/A cells
\providecommand{\XeLaTeX}{X\lower.5ex\hbox{\kern-0.15em\reflectbox{E}}\kern-0.1em\LaTeX}
\newcommand{\tXeLaTeX}{\XeLaTeX\index{XeLaTeX@\protect\XeLaTeX}}
% \index{\texttt{\textbackslash xyz}@\hangleft{\texttt{\textbackslash}}\texttt{xyz}}
\newcommand{\tuftebs}{\symbol{'134}}% a backslash in tt type in OT1/T1
\newcommand{\doccmdnoindex}[2][]{\texttt{\tuftebs#2}}% command name -- adds backslash automatically (and doesn't add cmd to the index)
\newcommand{\doccmddef}[2][]{%
  \hlred{\texttt{\tuftebs#2}}\label{cmd:#2}%
  \ifthenelse{\isempty{#1}}%
    {% add the command to the index
      \index{#2 command@\protect\hangleft{\texttt{\tuftebs}}\texttt{#2}}% command name
    }%
    {% add the command and package to the index
      \index{#2 command@\protect\hangleft{\texttt{\tuftebs}}\texttt{#2} (\texttt{#1} package)}% command name
      \index{#1 package@\texttt{#1} package}\index{packages!#1@\texttt{#1}}% package name
    }%
}% command name -- adds backslash automatically
\newcommand{\doccmd}[2][]{%
  \texttt{\tuftebs#2}%
  \ifthenelse{\isempty{#1}}%
    {% add the command to the index
      \index{#2 command@\protect\hangleft{\texttt{\tuftebs}}\texttt{#2}}% command name
    }%
    {% add the command and package to the index
      \index{#2 command@\protect\hangleft{\texttt{\tuftebs}}\texttt{#2} (\texttt{#1} package)}% command name
      \index{#1 package@\texttt{#1} package}\index{packages!#1@\texttt{#1}}% package name
    }%
}% command name -- adds backslash automatically
\newcommand{\docopt}[1]{\ensuremath{\langle}\textrm{\textit{#1}}\ensuremath{\rangle}}% optional command argument
\newcommand{\docarg}[1]{\textrm{\textit{#1}}}% (required) command argument
\newenvironment{docspec}{\begin{quotation}\ttfamily\parskip0pt\parindent0pt\ignorespaces}{\end{quotation}}% command specification environment
\newcommand{\docenv}[1]{\texttt{#1}\index{#1 environment@\texttt{#1} environment}\index{environments!#1@\texttt{#1}}}% environment name
\newcommand{\docenvdef}[1]{\hlred{\texttt{#1}}\label{env:#1}\index{#1 environment@\texttt{#1} environment}\index{environments!#1@\texttt{#1}}}% environment name
\newcommand{\docpkg}[1]{\texttt{#1}\index{#1 package@\texttt{#1} package}\index{packages!#1@\texttt{#1}}}% package name
\newcommand{\doccls}[1]{\texttt{#1}}% document class name
\newcommand{\docclsopt}[1]{\texttt{#1}\index{#1 class option@\texttt{#1} class option}\index{class options!#1@\texttt{#1}}}% document class option name
\newcommand{\docclsoptdef}[1]{\hlred{\texttt{#1}}\label{clsopt:#1}\index{#1 class option@\texttt{#1} class option}\index{class options!#1@\texttt{#1}}}% document class option name defined
\newcommand{\docmsg}[2]{\bigskip\begin{fullwidth}\noindent\ttfamily#1\end{fullwidth}\medskip\par\noindent#2}
\newcommand{\docfilehook}[2]{\texttt{#1}\index{file hooks!#2}\index{#1@\texttt{#1}}}
\newcommand{\doccounter}[1]{\texttt{#1}\index{#1 counter@\texttt{#1} counter}}

% Generates the index
\usepackage{makeidx}
\makeindex

\begin{document}

% Front matter
\frontmatter

% r.3 full title page
\maketitle


% v.4 copyright page
\newpage
\begin{fullwidth}
~\vfill
\thispagestyle{empty}
\setlength{\parindent}{0pt}
\setlength{\parskip}{\baselineskip}
Copyright \copyright\ \monthyear

\par\smallcaps{Published by \thanklesspublisher}
\par\smallcaps{www.hpcforhealth.ca}

\par\textit{First printing, \monthyear}
\end{fullwidth}

% r.9 introduction
\cleardoublepage
\chapter*{Executive Summary}

Staggeringly brillint summary goes here.

%%
% Start the main matter (normal chapters)
\mainmatter

\chapter{Human Health Research in the Era of Genomics}
\label{ch:health-data-at-scale}

\newthought{Next-generation genomics and electronic medical records} have 
become ubiquitous almost simultaneously, opening completely new windows onto the field
of human health research.  

\begin{figure}
 \includegraphics{cumulative_genomes.pdf}
  \caption[Cumulative number of human genomes sequenced]{The cumulative number of human genomes sequenced over the past 15 years,
    data from \protect\citep{stephens2015big}.  New technologies and data types have caused a inflection in the exponential rate
    of data growth which continues today.}
  \label{fig:exponential-growth}
\end{figure}

In genomics, an exponentially-growing number of human genomes are
being sequenced, making enormous amounts of evidence on the heredity
of human disease states at previously inaccessible levels of detail
(Fig~\ref{fig:exponential-growth}).  Data rates are only increasing as
new devices become available, and new types of data are being
generated; for instance, RNA sequencing allows us to go beyond
simply sequencing ``the'' genome of an individual and instead measure
the gene products being expressed in particular cells at particular
times, giving us insight into not just predispositions but the precise
disease state of cells over time.

Meanwhile, information technology elsewhere 
traditional design of front matter.  First, the pages in front matter are
traditionally numbered with lowercase roman numerals (\eg, i, ii, iii,
iv,~\ldots).  Second, the front matter page numbering sequence is usually
separate from the main matter page numbering.  That is, the page numbers
restart at 1 when the main matter begins.  In contrast, Tufte has
enumerated his pages with arabic numerals that share the same page counting
sequence as the main matter.  

There are also some variations in design across Tufte's four books.  The
page opposite the full title page (labeled ``frontispiece'' in the above
table) has different content in each of the books.  In \VDQI, this page is
blank; in \EI and \VE, this page holds a frontispiece; and in \BE, this
page contains three epigraphs.

The dedication appears on page~6 in \vdqi (opposite the introduction), and
is placed on its own spread in the other books.  In \ve, an epigraph shares
the spread with the opening page of the introduction.

None of the page numbers (folios) of the front matter are expressed except in
\be, where the folios start to appear on the dedication page.

\newthought{The full title page} of each of the books varies slightly in
design.  In all the books, the author's name appears at the top of the
page, the title it set just above the center line, and the publisher is
printed along the bottom margin.  Some of the differences are outlined in
the following table.


\section{Headings}\label{sec:headings1}\index{headings}

Tufte's books include the following heading levels: parts,
chapters,\sidenote{Parts and chapters are defined for the \texttt{tufte-book}
class only.}  sections, subsections, and paragraphs.  Not defined by default
are: sub-subsections and subparagraphs.

\paragraph{Paragraph} Paragraph headings (as shown here) are introduced by
italicized text and separated from the main paragraph by a bit of space.

\section{Environments}

The following characteristics define the various environments:


\chapter{HPC4Health --- a Made-in-Ontario Collaboration}
\label{ch:hpc4health}

The \TL document classes define a style similar to the
style Edward Tufte uses in his books and handouts.  Tufte's style is known
for its extensive use of sidenotes, tight integration of graphics with
text, and well-set typography.  This document aims to be at once a
demonstration of the features of the \TL document classes
and a style guide to their use.

\begin{figure}
  \includegraphics{HPC4Health_1.pdf}
  \caption[HPC4Health Architecture Diagram]{The HPC4Health Architecture}
  \label{fig:hpc4health-architecture}
\end{figure}


If you need more than two levels of section headings, you'll have to define
them yourself at the moment; there are no pre-defined styles for anything below
a \Verb|\subsection|.  As Bringhurst points out in \textit{The Elements of
Typographic Style}, you should ``use as many levels of
headings as you need: no more, and no fewer.''

The \TL classes will emit an error if you try to use
\linebreak\Verb|\subsubsection| and smaller headings.

% let's start a new thought -- a new section
\newthought{In his later books}, Tufte
starts each section with a bit of vertical space, a non-indented paragraph,
and sets the first few words of the sentence in \textsc{small caps}.  To
accomplish this using this style, use the \doccmddef{newthought} command:
\begin{docspec}
  \doccmd{newthought}\{In his later books\}, Tufte starts\ldots
\end{docspec}


\section{Sidenotes}\label{sec:sidenotes}
One of the most prominent and distinctive features of this style is the
extensive use of sidenotes.  There is a wide margin to provide ample room
for sidenotes and small figures.  Any \doccmd{footnote}s will automatically
be converted to sidenotes.\footnote{This is a sidenote that was entered
using the \texttt{\textbackslash footnote} command.}  If you'd like to place ancillary
information in the margin without the sidenote mark (the superscript
number), you can use the \doccmd{marginnote} command.\marginnote{This is a
margin note.  Notice that there isn't a number preceding the note, and
there is no number in the main text where this note was written.}

The specification of the \doccmddef{sidenote} command is:
\begin{docspec}
  \doccmd{sidenote}[\docopt{number}][\docopt{offset}]\{\docarg{Sidenote text.}\}
\end{docspec}

Both the \docopt{number} and \docopt{offset} arguments are optional.  If you
provide a \docopt{number} argument, then that number will be used as the
sidenote number.  It will change of the number of the current sidenote only and
will not affect the numbering sequence of subsequent sidenotes.

\chapter{The H4H Network --- Building on Strengths}

\label{ch:hpc4health}
\begin{figure}
  \includegraphics{H4HNetwork.pdf}
  \caption[The H4H Network]{The H4H Network}
  \label{fig:hpc4health-architecture}
\end{figure}


%%
% The back matter contains appendices, bibliographies, indices, glossaries, etc.


\backmatter

\bibliography{h4h_network}
\bibliographystyle{plainnat}


\printindex

\end{document}

