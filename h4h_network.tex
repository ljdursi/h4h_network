\documentclass[twoside,symmetric,sfsidenotes,notoc]{tufte-book}

\hypersetup{colorlinks}% uncomment this line if you prefer colored hyperlinks (e.g., for onscreen viewing)

%%
% Book metadata
\title{The HPC4Health Network}
\subtitle{Building an Ontario-wide platform for human health data}
\author{HPC4Health: High-Performance Data and Computing for Health Care}
\publisher{HPC4Health}

%%
% If they're installed, use Bergamo and Chantilly from www.fontsite.com.
% They're clones of Bembo and Gill Sans, respectively.
%\IfFileExists{bergamo.sty}{\usepackage[osf]{bergamo}}{}% Bembo
%\IfFileExists{chantill.sty}{\usepackage{chantill}}{}% Gill Sans

%\usepackage{microtype}

%%
% Just some sample text
\usepackage{lipsum}

%%
% For nicely typeset tabular material
\usepackage{booktabs}

%%
% For graphics / images
\usepackage{graphicx}
\setkeys{Gin}{width=\linewidth,totalheight=\textheight,keepaspectratio}
\graphicspath{{graphics/}}

% The fancyvrb package lets us customize the formatting of verbatim
% environments.  We use a slightly smaller font.
\usepackage{fancyvrb}
\fvset{fontsize=\normalsize}

%%
% Prints argument within hanging parentheses (i.e., parentheses that take
% up no horizontal space).  Useful in tabular environments.
\newcommand{\hangp}[1]{\makebox[0pt][r]{(}#1\makebox[0pt][l]{)}}

%%
% Prints an asterisk that takes up no horizontal space.
% Useful in tabular environments.
\newcommand{\hangstar}{\makebox[0pt][l]{*}}

%%
% Prints a trailing space in a smart way.
\usepackage{xspace}

%%
% Some shortcuts for Tufte's book titles.  The lowercase commands will
% produce the initials of the book title in italics.  The all-caps commands
% will print out the full title of the book in italics.
\newcommand{\vdqi}{\textit{VDQI}\xspace}
\newcommand{\ei}{\textit{EI}\xspace}
\newcommand{\ve}{\textit{VE}\xspace}
\newcommand{\be}{\textit{BE}\xspace}
\newcommand{\VDQI}{\textit{The Visual Display of Quantitative Information}\xspace}
\newcommand{\EI}{\textit{Envisioning Information}\xspace}
\newcommand{\VE}{\textit{Visual Explanations}\xspace}
\newcommand{\BE}{\textit{Beautiful Evidence}\xspace}

\newcommand{\TL}{Tufte-\LaTeX\xspace}

% Prints the month name (e.g., January) and the year (e.g., 2008)
\newcommand{\monthyear}{%
  \ifcase\month\or January\or February\or March\or April\or May\or June\or
  July\or August\or September\or October\or November\or
  December\fi\space\number\year
}


% Prints an epigraph and speaker in sans serif, all-caps type.
\newcommand{\openepigraph}[2]{%
  %\sffamily\fontsize{14}{16}\selectfont
  \begin{fullwidth}
  \sffamily\large
  \begin{doublespace}
  \noindent\allcaps{#1}\\% epigraph
  \noindent\allcaps{#2}% author
  \end{doublespace}
  \end{fullwidth}
}

% Inserts a blank page
\newcommand{\blankpage}{\newpage\hbox{}\thispagestyle{empty}\newpage}

\usepackage{units}

% Typesets the font size, leading, and measure in the form of 10/12x26 pc.
\newcommand{\measure}[3]{#1/#2$\times$\unit[#3]{pc}}

% Macros for typesetting the documentation
\newcommand{\hlred}[1]{\textcolor{Maroon}{#1}}% prints in red
\newcommand{\hangleft}[1]{\makebox[0pt][r]{#1}}
\newcommand{\hairsp}{\hspace{1pt}}% hair space
\newcommand{\hquad}{\hskip0.5em\relax}% half quad space
\newcommand{\TODO}{\textcolor{red}{\bf TODO!}\xspace}
\newcommand{\ie}{\textit{i.\hairsp{}e.}\xspace}
\newcommand{\eg}{\textit{e.\hairsp{}g.}\xspace}
\newcommand{\na}{\quad--}% used in tables for N/A cells
\providecommand{\XeLaTeX}{X\lower.5ex\hbox{\kern-0.15em\reflectbox{E}}\kern-0.1em\LaTeX}
\newcommand{\tXeLaTeX}{\XeLaTeX\index{XeLaTeX@\protect\XeLaTeX}}
% \index{\texttt{\textbackslash xyz}@\hangleft{\texttt{\textbackslash}}\texttt{xyz}}
\newcommand{\tuftebs}{\symbol{'134}}% a backslash in tt type in OT1/T1
\newcommand{\doccmdnoindex}[2][]{\texttt{\tuftebs#2}}% command name -- adds backslash automatically (and doesn't add cmd to the index)
\newcommand{\doccmddef}[2][]{%
  \hlred{\texttt{\tuftebs#2}}\label{cmd:#2}%
  \ifthenelse{\isempty{#1}}%
    {% add the command to the index
      \index{#2 command@\protect\hangleft{\texttt{\tuftebs}}\texttt{#2}}% command name
    }%
    {% add the command and package to the index
      \index{#2 command@\protect\hangleft{\texttt{\tuftebs}}\texttt{#2} (\texttt{#1} package)}% command name
      \index{#1 package@\texttt{#1} package}\index{packages!#1@\texttt{#1}}% package name
    }%
}% command name -- adds backslash automatically
\newcommand{\doccmd}[2][]{%
  \texttt{\tuftebs#2}%
  \ifthenelse{\isempty{#1}}%
    {% add the command to the index
      \index{#2 command@\protect\hangleft{\texttt{\tuftebs}}\texttt{#2}}% command name
    }%
    {% add the command and package to the index
      \index{#2 command@\protect\hangleft{\texttt{\tuftebs}}\texttt{#2} (\texttt{#1} package)}% command name
      \index{#1 package@\texttt{#1} package}\index{packages!#1@\texttt{#1}}% package name
    }%
}% command name -- adds backslash automatically
\newcommand{\docopt}[1]{\ensuremath{\langle}\textrm{\textit{#1}}\ensuremath{\rangle}}% optional command argument
\newcommand{\docarg}[1]{\textrm{\textit{#1}}}% (required) command argument
\newenvironment{docspec}{\begin{quotation}\ttfamily\parskip0pt\parindent0pt\ignorespaces}{\end{quotation}}% command specification environment
\newcommand{\docenv}[1]{\texttt{#1}\index{#1 environment@\texttt{#1} environment}\index{environments!#1@\texttt{#1}}}% environment name
\newcommand{\docenvdef}[1]{\hlred{\texttt{#1}}\label{env:#1}\index{#1 environment@\texttt{#1} environment}\index{environments!#1@\texttt{#1}}}% environment name
\newcommand{\docpkg}[1]{\texttt{#1}\index{#1 package@\texttt{#1} package}\index{packages!#1@\texttt{#1}}}% package name
\newcommand{\doccls}[1]{\texttt{#1}}% document class name
\newcommand{\docclsopt}[1]{\texttt{#1}\index{#1 class option@\texttt{#1} class option}\index{class options!#1@\texttt{#1}}}% document class option name
\newcommand{\docclsoptdef}[1]{\hlred{\texttt{#1}}\label{clsopt:#1}\index{#1 class option@\texttt{#1} class option}\index{class options!#1@\texttt{#1}}}% document class option name defined
\newcommand{\docmsg}[2]{\bigskip\begin{fullwidth}\noindent\ttfamily#1\end{fullwidth}\medskip\par\noindent#2}
\newcommand{\docfilehook}[2]{\texttt{#1}\index{file hooks!#2}\index{#1@\texttt{#1}}}
\newcommand{\doccounter}[1]{\texttt{#1}\index{#1 counter@\texttt{#1} counter}}

% Generates the index
\usepackage{makeidx}
\makeindex

\begin{document}

% Front matter
\frontmatter

% r.3 full title page
\maketitle


% v.4 copyright page
\newpage
\begin{fullwidth}
~\vfill
\thispagestyle{empty}
\setlength{\parindent}{0pt}
\setlength{\parskip}{\baselineskip}
Copyright \copyright\ \monthyear
\par\smallcaps{Published by \thanklesspublisher}
\par\smallcaps{www.hpcforhealth.ca}
\end{fullwidth}

% r.9 introduction
\cleardoublepage
\chapter*{Executive Summary}

Staggeringly brillint summary goes here.

%%
% Start the main matter (normal chapters)
\mainmatter

\chapter{Human Health Research in the Era of Genomics}
\label{ch:health-data-at-scale}

\newthought{Next-generation genomics and electronic medical records} have 
become ubiquitous almost simultaneously, opening completely new windows onto the field
of human health research.  

\begin{figure}
 \includegraphics{cumulative_genomes.pdf}
  \caption[Cumulative number of human genomes sequenced]{The cumulative number of human genomes sequenced over the past 15 years,
    data from \protect\citep{stephens2015big}.  New technologies and data types have caused a inflection in the exponential rate
    of data growth which continues today.}
  \label{fig:exponential-growth}
\end{figure}

In genomics, an exponentially-growing number of human genomes are
being sequenced, making enormous amounts of evidence on the heredity
of human disease states at previously inaccessible levels of detail
(Fig~\ref{fig:exponential-growth}).  Data rates are only increasing as
new devices become available, and new types of data are being
generated; for instance, RNA sequencing allows us to go beyond
simply sequencing ``the'' genome of an individual and instead measure
the gene products being expressed in particular cells at particular
times, giving us insight into not just predispositions but the precise
disease state of cells over time.

Until now, the bulk of this genomic data has come from research projects ---
focused, generally short-term efforts to answer specific questions using 
genomic information. However, two changes in the practice of medicine are
poised to radically increase the rate of genomic and human health data creation
in Canada.

First, genomic medicine\footnote{Genomic Medicine, as defined by the NIH's
National Human Genome Research Institute: ``An emerging medical discipline
that involves using genomic information about an individual as part of their
clinical care (\textit{e.g.}, for diagnostic or therapeutic decision-making)''} is beginning
to enter the standard of care, initially in the cases of hard-to-diagnose
rare diseases or recurrent cancers, and starting to displace smaller
and more limited genetic tests in other areas.  The sheer scale of clinical medicine 
--- in Canada, hospital spending alone is nearly seventy times the research funding
budget of the CIHR --- means that as adoption increases, clinical
genomic data creation will rapidly outpace that of research genomic data.

Second, information technology advances elsewhere in healthcare
has led to the rapid adoption of Electronic Medical Record systems\footnote{
Electronic Medical Records, or EMRs, are digital version of paper chapters
in hospitals or doctors offices; Electronic Health Records, or EHRs, are
more integrated systems combining information across practices and institutions.
EMRs are mature and growing in adoption rapidly, while EHR systems are still
some time from being common.}, describing a patient's condition, tests, and
treatment in detail and at least partially in machine-readable form.

The rapid growth of genomic data volumes and the increasing depth and detail
of clinical data present in EMRs offers enormous promise for human health research,
with insights into both basic biology and to future treatments.  The joint analyses
of clinical and phenotypic health record data along with genomic information 
about the patient and their disease offers unprecented opportunity for researchers
to connect genetic predisposition, treatments, and outcomes, allowing the
development of national, truly precision, medicine practices.

But making use of this data in an era of rapid growth raises multiple challenges.
On the physical infrastructure side, simply making available the storage 
resources to capture and archive the onrush of data is a challenge, along with
providing the computational power to perform increasingly sophisticated analyses.
Architecting, building, and maintaining these systems, particularly tuned to
the needs of health research, requires a specialized approach.

Human infrastructure is also required.  Making productive use of the 
data means ensuring that the expertise exists and is available to for interpretation,
and that those experts are continuously kept up to date on the new types of data
and new techniques for analysis.  This requires funding, ongoing training,
and opportunities for professional recognition and growth of the experts, whether
they be bioinformaticians, computational biologists, or systems administrators.

Finally, the unique challenges of dealing with health data means that the duty
of care to patients to zealously protect the security and their privacy is
paramount; sophisticated and enforcable policies around data governance and consent
are required, along with international best practices around security and monitoring.

\chapter{HPC4Health --- a Made-in-Ontario Collaboration}
\label{ch:hpc4health}

The \TL document classes define a style similar to the
style Edward Tufte uses in his books and handouts.  Tufte's style is known
for its extensive use of sidenotes, tight integration of graphics with
text, and well-set typography.  This document aims to be at once a
demonstration of the features of the \TL document classes
and a style guide to their use.

\begin{figure}
  \includegraphics{HPC4Health_1.pdf}
  \caption[HPC4Health Architecture Diagram]{The HPC4Health Architecture}
  \label{fig:hpc4health-architecture}
\end{figure}


If you need more than two levels of section headings, you'll have to define
them yourself at the moment; there are no pre-defined styles for anything below
a \Verb|\subsection|.  As Bringhurst points out in \textit{The Elements of
Typographic Style}, you should ``use as many levels of
headings as you need: no more, and no fewer.''

The \TL classes will emit an error if you try to use
\linebreak\Verb|\subsubsection| and smaller headings.

% let's start a new thought -- a new section
\newthought{In his later books}, Tufte
starts each section with a bit of vertical space, a non-indented paragraph,
and sets the first few words of the sentence in \textsc{small caps}.  To
accomplish this using this style, use the \doccmddef{newthought} command:
\begin{docspec}
  \doccmd{newthought}\{In his later books\}, Tufte starts\ldots
\end{docspec}


\section{Sidenotes}\label{sec:sidenotes}
One of the most prominent and distinctive features of this style is the
extensive use of sidenotes.  There is a wide margin to provide ample room
for sidenotes and small figures.  Any \doccmd{footnote}s will automatically
be converted to sidenotes.\footnote{This is a sidenote that was entered
using the \texttt{\textbackslash footnote} command.}  If you'd like to place ancillary
information in the margin without the sidenote mark (the superscript
number), you can use the \doccmd{marginnote} command.\marginnote{This is a
margin note.  Notice that there isn't a number preceding the note, and
there is no number in the main text where this note was written.}

The specification of the \doccmddef{sidenote} command is:
\begin{docspec}
  \doccmd{sidenote}[\docopt{number}][\docopt{offset}]\{\docarg{Sidenote text.}\}
\end{docspec}

Both the \docopt{number} and \docopt{offset} arguments are optional.  If you
provide a \docopt{number} argument, then that number will be used as the
sidenote number.  It will change of the number of the current sidenote only and
will not affect the numbering sequence of subsequent sidenotes.

\chapter{The H4H Network --- Building on Strengths}
\label{ch:hpc4health_network}

\begin{figure}
  \includegraphics{H4HNetwork.pdf}
  \caption[The H4H Network]{The H4H Network}
  \label{fig:hpc4health_network}
\end{figure}


%%
% The back matter contains appendices, bibliographies, indices, glossaries, etc.


\backmatter

\bibliography{h4h_network}
\bibliographystyle{plainnat}


\printindex

\end{document}

